\documentclass{article}
\usepackage{graphicx}
\usepackage{float}
\usepackage[utf8]{inputenc}

\title{\textbf{Simulation Proposal}\\CS4ZP6}
\author{Aaska Shah\\Kerala Brendon\\Nolan Slade\\Vyome Kishore}
\date{\today}

\begin{document}
\maketitle

\section*{Overview}

Proposed in this document are two candidates for the McMaster Decision Science Laboratory's first virtual reality-based economics research simulations. Similar to the existing simulation, the crate game, each candidate is designed around the user repetitively completing a basic task with the goal of maximizing earnings that are awarded on every completed iteration of the task. The simulations are structured such that their lifetimes are separated into discrete time periods, \textit{days}, in which the user's ability to perform the repetitive task is hindered by an \textit{impairment(s)}. The user will be given the option to reduce the effects of such an impairment by receiving a \textit{treatment} - such treatments can be paid for, or received for free after waiting a predetermined duration.
\\
\\
Each candidate will feature both a virtual environment designed within Unity and HTC Vive support to offer a truly immersive simulation experience to the participant.


\section*{Candidate One: Liquid Transport}

In this simulation, the participant will be required to repeatedly transport a volume of liquid between a source and destination using a single hand-carried vessel. Their goal will be to maximize the total volume of liquid that successfully reaches the destination.


\subsection*{Environment, Task and Reward Breakdown}

The source and destination that the user will interact with will be placed on opposing sides of the virtual environment such that the user will be required to both move laterally as well as rotate to complete the required task.

\begin{figure}[H]
    \includegraphics[width=150pt]{liquid_transport_environment.png}
    \centering
    \caption{Proposed virtual environment for the Liquid Transport simulation.}
\end{figure}

The source will take the form of a standard faucet, and the destination will be presented as a circular sink. The radius of the sink will be chosen such that an unimpaired user would be able to successfully pour the liquid contained within the vessel into the sink with no, or at most a negligible amount, of spillage. The vessel available to the participant will initially be placed beside the source, and will take the form of a vase.
\\
\\
First, the participant will be required to pick up the initially empty vase. Then, they must proceed to hold the vase underneath the running source until the volume of liquid is at the level desired - we propose a transparent vase to facilitate this. After filling the vase to their liking, the participant will be required to hold onto the filled vase, turn around, and walk a short distance to reach the destination sink. The final goal will be to minimize spillage such that the volume of liquid successfully poured into the sink is as great as possible.
\\
\\
We propose that the participant is rewarded per unit of liquid successfully poured into the sink, for example, \$0.01 per mL. This reward function will be constant across the lifetime of the simulation.

\section*{Candidate Two: Shape Sorting}

The second simulation will provide the participant with a set of three-dimensional shapes, and task them to sort them into separate containers by passing them through a filter that only allows one particular shape through. The aim of the participant will be to maximize the total number of shapes successfully sorted into their respective containers.

\subsection*{Environment, Task and Reward Breakdown}

This simulation will feature a virtual environment that does not require the participant to move as much as in candidate one. Instead, the participant will initially be positioned between the source bin and the shape-specific containers. 

\begin{figure}[H]
    \includegraphics[width=180pt]{shape_sorting_environment.png}
    \centering
    \caption{Proposed virtual environment for candidate two: shape sorting.}
\end{figure}

Inside the source bin, the participant will find a large amount of shapes with their respective geometries matching exactly one shape-specific container. As discussed, it is the participant's goal to pick up shapes one-at-a-time, and place them into their respective bins by passing them through the correct filter. 
\\
\\
Each destination bin filter will be large enough that an unimpaired user will experience minimal difficulty in manipulating the shape such that it may pass through.

\begin{figure}[H]
    \includegraphics[width=180pt]{shape_shorting_filter.png}
    \centering
    \caption{An hypothetical set of destination bins with their respective filters.}
\end{figure}
\newpage
Care will be taken to generate geometries that are mutually exclusive, that is, that no two shapes may pass through the same filter.
\\
\\
We propose to reward the user with a constant dollar figure per shape correctly deposited into a destination bin, for example, \$1.00 per shape.

\section*{Impairment Possibilities}

We propose up to two impairments for both candidates that can be experienced by the participant either in tandem or separately: a physical impairment, and a visual impairment.
\\
\\
The goal of the first impairment is to simulate the participant having poor motor skills through the modification of the physical properties of certain items within the simulation environment. More specifically, for candidate one, we propose to modify the effects of gravity on the vase such that it is more likely that the user will spill some of its contents both en route to the destination sink as well as during the pouring process. Similarly, for candidate two, we propose simulating the participant losing hand strength by making the shapes slip out of their grip.
\\
\\
The second impairment we propose is to simulate poor vision through the introduction of a fog to each environment. With such an impairment in simulation one, it would be less trivial to determine: 1. the fill-level of the vessel, 2. the locations of the source and sink, and 3. the exact boundaries of the sink. In simulation two, it would become more difficult to determine the proper destination bin for a given shape.
\\
\\
For both candidates, we prefer the use of a veil or fog to more complex visual impairments due to the fact it is less likely that it will cause a participant to experience motion sickness.


\end{document}
