\documentclass{article}
\usepackage[utf8]{inputenc}
\usepackage{listings}
\title{User Manual: Configuration File}
\author{Aaska Shah\\Kerala Brendon\\Nolan Slade\\Vyome Kishore}
\date{April 2019}

\usepackage{natbib}
\usepackage{graphicx}

\begin{document}

\maketitle

\section*{Overview}
This document outlines how to modify the simulation configuration file in order to achieve the desired experiment format. The application expects such a file to be named \textit{sim\_config.txt}, and for it to be located in the \textit{Water\_Sim\_x.x\slash simulation\_one\_Data\slash InputData} directory. Failing to follow this naming convention or moving the file out of this location will result in the application being unable to start up properly. The configuration file supports a number of keyword-value pairs that can be easily combined to define the temporal structure of the simulation as well as auxiliary parameters. 

\section*{Keywords}
\subsection*{Simulation}
This section includes information pertaining to high-level details of the simulation. This keyword is not tabbed; each of the following four keywords are tabbed once.

\subsubsection*{Name} For user convenience - offers a simple way of differentiating multiple configuration file setups. No restrictions on value. 

%\noindent \textbf{Output}: How the experimenter wants the data from the simulation to be produce\newline
%\indent -Possible Values: (both can be used, separated by a comma)\newline
%\indent\indent .txt \newline
%\indent\indent database \newline
%
%\noindent Example:
%\begin{lstlisting}
%Output:.txt,database
%\end{lstlisting}

\subsubsection*{Description} Similar to the above: offers a way to describe a given simulation configuration for documentation purposes. No restrictions on value. 
\pagebreak\subsubsection*{Instructions} Toggles whether or not instructions will appear throughout the experiment. These include both the \textit{Day Zero} tutorial and pre-treatment instruction sets.
\newline \indent Possible values: \newline
\indent\indent enabled \newline
\indent\indent disabled

\subsubsection*{Sound} Disables or enables all sound effects within the scene, including water flow and countdowns, among others.
\newline \indent Possible values: \newline
\indent\indent enabled \newline
\indent\indent disabled

\subsubsection*{Sample Simulation Configuration:}
\begin{lstlisting}
Simulation
    Name:April 2019 configuration.
    Description:Stronger impairments.
    Instructions:enabled
    Sound:enabled
\end{lstlisting} 

\subsection*{Tutorial}
This keyword refers to the \textit{Day Zero} tutorial preceding the main experiment. During this stage of the simulation, the user is required to earn a certain amount of money before being allowed to move on to the first paid day. Optionally, \textit{Day Zero} can be split into two portions: one unimpaired portion, and one impaired portion. The impaired portion is preceded by an instruction set. The second portion will only take place if the \textit{Impairment} keyword is used when configuring the tutorial. Usage of the \textit{Impairment} keyword is described later in this document. The \textit{Tutorial} keyword is not tabbed, while the \textit{Score}, \textit{ImpairedScore}, and \textit{Impairment} keywords are tabbed once. Sub-keywords for the optional second portion impairment, \textit{Strength} and \textit{Type}, should be tabbed three times.

\subsubsection*{Score} Sets the amount that the participant will need to earn to pass the tutorial day. If not specified, the default is \textbf{\$150.00}. \newline
\indent Possible values: \newline
\indent\indent Any decimal number

\pagebreak\subsubsection*{ImpairedScore} Sets the amount that the participant will need to earn to pass the optional second portion of the tutorial day. If not specified, the default is also \textbf{\$150.00}. \newline
\indent Possible values: \newline
\indent\indent Any decimal number

\subsubsection*{Sample tutorial configuration:}
\begin{lstlisting}
Tutorial
    Score:175
    ImpairedScore:125
    Impairment
        Type:Physical/Shake
        Strength:50%
\end{lstlisting}

\subsection*{Day}
Signifies a new day to be included in the experiment. Each day may include the following sub-keywords: \textit{Duration}, \textit{Impairment}, and \textit{Treatment}. The \textit{Day} keyword is not tabbed, and each associated sub-keyword is tabbed once.

\subsubsection*{Duration}
This keyword refers to how long each day will last. This field is \textbf{mandatory}. \newline
\indent Possible Values: \newline
\indent\indent minutes:seconds, where both components must be positive integers

\subsubsection*{WaterValue}
This keyword allows for the payout per droplet to be adjusted. This is not a required field, and by default it is set to 1. \newline
\indent Possible Values: \newline
\indent\indent Any decimal number

\subsubsection*{Impairment}
This keyword will determine which impairments will be in effect during the day. There can be one or more impairments imposed on a single day, and each one requires its own \textit{Impairment} keyword. \textit{Impairment} is tabbed once. Associated sub-keywords, \textit{Type} and \textit{Strength}, are tabbed twice. \newline

\subsubsubsection{\textbf{Type}} \newline
\indent Possible values: \newline
\indent\indent Physical/Shake \newline
\indent\indent Visual/Fog \newline
%\indent\indent Physical/Speed\newline
%\indent\indent Physical/Gravity\newline


\subsubsubsection{\textbf{Strength}} \newline
\indent Possible values: \newline
\indent\indent Any integer between 0-100 followed by a \% sign \newline
    
%\noindent Example:
%\begin{lstlisting}
%    Impairment
%        Type:Physical/Speed
%        Strength:75%
%    Impairment
%        Type:Visual/Fog
%        Strength:5%
%\end{lstlisting}

\subsubsection*{Treatment}
This keyword is used to specify which treatment options will be available to the participant on the given day. The cost of obtaining a treatment follows the functional form of $C(c - bT + aT^2)$, where by default, the values of a, b, and c are 1\slash day length in minutes, 2, and day length in minutes, respectively. The functional form is consistent across both pay-style treatments (where cost is in dollars), and wait-style treatments (where cost is in seconds). When the \textit{Wait} or \textit{Cost} keywords are included, their respective \textbf{C} values are \textbf{mandatory}; and, if all three of their respective a, b, and c values are excluded entirely, they will be set to default. 

Typically, a treatment alleviates 100\% of all active impairments, with 100\% certainty. If desired, these two values can be modified using the \textit{Effectiveness} keyword, with sub-keywords \textit{Effect}, and \textit{Probability}, respectively. These two fields are not mandatory and will default to 100\% each if excluded.

\textit{Treatment} is tabbed once while sub-keywords \textit{Effectiveness}, \textit{Wait}, and \textit{Cost} are tabbed twice. \textit{C}, \textit{a}, \textit{b}, \textit{c}, \textit{Effect}, and \textit{Probability} are tabbed three times. \newline

%\subsubsection*{Treatment}
%This keyword is used to specify which treatment options will be available to the participant on the given day. When the \textit{Wait} or \textit{Cost} keywords are included, their respective \textbf{C} values are \textbf{mandatory} values.  \textit{Treatment} is tabbed once while sub-keywords \textit{Wait}, \textit{Cost}, \textit{Certainty} and \textit{Effectiveness} are tabbed twice. \textit{C}, \textit{a}, \textit{b}, \textit{c} are tabbed three times, as they are specific to exactly one treatment.

%\subsubsubsection{\textbf{Certainty}} \newline
%\indent Possible values:\newline
%\indent\indent Any integer between 0-100 followed by a \% sign \newline

%\noindent Example: 
%\begin{lstlisting}
%    Treatment
%        Certainty:80%
%\end{lstlisting}

\subsubsubsection{\textbf{C}} \newline
\indent Possible values: \newline
\indent\indent Any decimal number \newline\newline
\subsubsubsection{\indent\textbf{a, b, c}} \newline
\indent  Possible values: \newline
\indent\indent  Any decimal number \newline
\indent\indent  default \newline\newline
\subsubsubsection{\indent\textbf{Effect}} \newline
\indent  Possible values: \newline
\indent\indent  Any decimal number, followed by a \% sign \newline\newline
\subsubsubsection{\indent\textbf{Probability}} \newline
\indent  Possible values: \newline
\indent\indent  Any decimal number, followed by a \% sign \newline

%\noindent \textbf{Wait/Cost} \newline
%\noindent Example: Not a valid input
%\begin{lstlisting}
%    C:100
%    b:100
%    c:default
%\end{lstlisting}
%- "a" must be included, no option to omit only one of a, b, or c\newline

%\noindent \textbf{a, b, and c Included} \newline
%\noindent Example:
%\begin{lstlisting}
%    Treatment
%        Wait
%            C:100
%            a:default
%            b:default
%            c:10
%        Cost
%            C:56
%            a:34
%            b:54
%            c:default
% \end{lstlisting}   
% \noindent \textbf{a, b, and c Not Included} \newline
% \noindent Example:
% \begin{lstlisting}
%    Treatment
%        Wait
%            C:100
%        Cost
%            C:56
% \end{lstlisting} 
 
%\noindent \textbf{Effectiveness} 
%    
%\noindent Probability\newline
%\indent-Possible values:\newline
%\indent\indent Any whole/decimal number between 0-100 followed by a "\%" sign \newline
%\noindent Effect\newline
%\indent-Possible values:\newline
%\indent\indent Any whole/decimal number between 0-100 followed by a "\%" sign \newline    
    
\pagebreak\section*{Sample Configuration File}
\begin{lstlisting}
Simulation
    Name:Full sample configuration file
    Description:Sample 3-day (plus intro) configuration file  
    Instructions:enabled
    Sound:enabled
Tutorial
    Score:180
Day
    Duration:1:30
Day
    WaterValue:1.50
    Duration:1:30
    Impairment
        Type:Physical/Shake
        Strength:70%
Day
    WaterValue:1.25
    Duration:2:00
    Impairment
        Type:Visual/Fog
        Strength:75%
    Impairment
        Type:Physical/Shake
        Strength:50%
    Treatment
        Effectiveness
            Probability:50%
            Effect:90%
        Wait
            C:0.5
            a:0.1111
            b:3
            c:1.5
        Cost
            C:80
            a:default
            b:default
            c:default
\end{lstlisting}    

\end{document}
