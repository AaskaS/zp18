\documentclass{article}
\usepackage[utf8]{inputenc}
\usepackage{listings}
\title{User Manual: Configuration File}
\author{Aaska Shah}

\usepackage{natbib}
\usepackage{graphicx}

\begin{document}

\maketitle

\section*{Overview}
This document outlines how to modify the simulation configuration file in order to achieve the desired experiment format. The application expects such a file to be named \textit{sim\_config.txt}, and for it to be located in the \textit{Water\_Sim\_x.x\slash simulation\_one\_Data\slash InputData} directory. Failing to follow this naming convention or moving the file out of this location will result in the application being unable to start up properly. The configuration file supports a number of keyword-value pairs that can be easily combined to define the temporal structure of the simulation as well as auxiliary parameters. 

\section*{Keywords}
\subsection*{Simulation}
This section includes information pertaining to high-level details of the simulation. This keyword is not tabbed; each keyword under this is tabbed once.

\subsubsection*{Name} For user convenience - offers a simple way of differentiating multiple configuration file setups. No restrictions on value. 

%
%\noindent \textbf{Output}: How the experimenter wants the data from the simulation to be produce\newline
%\indent -Possible Values: (both can be used, separated by a comma)\newline
%\indent\indent .txt \newline
%\indent\indent database \newline
%
%\noindent Example:
%\begin{lstlisting}
%Output:.txt,database
%\end{lstlisting}

\subsubsection*{Description} Similar to the above; offers a way to describe a given simulation configuration for documentation purposes. No restrictions on value. 

\subsubsection*{Instructions} Toggles whether or not instructions will appear throughout the experiment. These include both the \textit{Day Zero} tutorial and pre-treatment instruction sets.
\newline \indent Possible values: \newline
\indent\indent enabled \newline
\indent\indent disabled

\subsubsection*{Sound} Disables or enables all sound effects within the scene, including water flow and countdowns, among others.
\newline \indent Possible values: \newline
\indent\indent enabled \newline
\indent\indent disabled

\subsubsection*{Sample Simulation Configuration:}
\begin{lstlisting}
Simulation
    Name:April 2019 configuration.
    Description:Stronger impairments.
    Instructions:enabled
    Sound:enabled
\end{lstlisting} 

\subsection*{Tutorial}
This keyword refers to the \textit{Day Zero} tutorial preceding the main experiment. During this portion of the simulation, the user is required to earn a certain amount of money before being allowed to move on to the rest of the days. The \textit{Tutorial} keyword is not tabbed, while the \textit{Score} keyword is tabbed once.

\subsubsection*{Score} Sets the amount that the participant will need to earn to pass the tutorial day. If not specified, the default is \textbf{\$150.00}. \newline
\indent Possible values: \newline
\indent\indent Any decimal number \newline

\subsubsection*{Sample tutorial configuration:}
\begin{lstlisting}
Tutorial
    Score:120
\end{lstlisting}

\subsection*{Day}
Signifies a new day to be included in the experiment. Each day may include the following sub-keywords: \textit{Duration}, \textit{Impairment}, and \textit{Treatment}. The \textit{Day} keyword is not tabbed, and each associated sub-keyword is tabbed once.

\subsubsection*{Duration}
This keyword refers to how long each day will last. This field is \textbf{mandatory}. \newline
\indent Possible Values: \newline
\indent\indent minutes:seconds, where both components must be positive integers \newline

\subsubsection*{Impairment}
This keyword will determine which impairments will be in effect during the day. There can be one or more impairments imposed on a single day; each one requires its own \textit{Impairment} keyword. \textit{Impairment} is tabbed once. Associated sub-keywords, \textit{Type} and \textit{Strength}, are tabbed twice. \newline

\subsubsubsection{\textbf{Type}} \newline
\indent Possible values: \newline
\indent\indent Physical/Shake \newline
\indent\indent Visual/Fog \newline
%\indent\indent Physical/Speed\newline
%\indent\indent Physical/Gravity\newline


\subsubsubsection{\textbf{Strength}} \newline
\indent Possible values: \newline
\indent\indent Any integer between 1-100 followed by a \% sign \newline
    
%\noindent Example:
%\begin{lstlisting}
%    Impairment
%        Type:Physical/Speed
%        Strength:75%
%    Impairment
%        Type:Visual/Fog
%        Strength:5%
%\end{lstlisting}

\subsubsection{Treatment}
This keyword will determine the amount of wait time and the amount of cost the treatment will have on the specified day. When \textbf{Wait} or \textbf{Cost} is included, the \textbf{C} is a \textbf{mandatory} value. The keywords \textbf{a}, \textbf{b}, \textbf{c} can either all be included or not included at all. There is also the option to include Wait, or Cost, or both.
\newline \textit{Treatment} is tabbed once while \textit{Wait}, \textit{Cost}, \textit{Certainty} and \textit{Effectiveness} are tabbed twice. \textit{C}, \textit{a}, \textit{b}, \textit{c}, \textit{Probability}, and \textit{Effect} are tabbed three times.\newline


\noindent \textbf{Certainty} \newline
\indent-Possible values:\newline
\indent\indent Any whole/decimal number between 0-100 followed by a "\%" sign \newline

\noindent Example: 
\begin{lstlisting}
    Treatment
        Certainty:80%
\end{lstlisting}


\noindent \textbf{Wait / Cost} \newline
\noindent C \newline
\indent -Possible Values:\newline
\indent\indent Any whole/decimal number\newline
\noindent a, b, and c\newline
\indent -Possible Values:\newline
\indent\indent Any whole/decimal numbers\newline
\indent\indent default\newline

\noindent \textbf{Wait/Cost} \newline
\noindent Example: Not a valid input
\begin{lstlisting}
    C:100
    b:100
    c:default
\end{lstlisting}
- "a" must be included, no option to omit only one of a, b, or c\newline


\noindent \textbf{a, b, and c Included} \newline
\noindent Example:
\begin{lstlisting}
    Treatment
        Wait
            C:100
            a:default
            b:default
            c:10
        Cost
            C:56
            a:34
            b:54
            c:default
 \end{lstlisting}   
 \noindent \textbf{a, b, and c Not Included} \newline
 \noindent Example:
 \begin{lstlisting}
    Treatment
        Wait
            C:100
        Cost
            C:56
 \end{lstlisting} 
 
\noindent \textbf{Effectiveness} 
    
\noindent Probability\newline
\indent-Possible values:\newline
\indent\indent Any whole/decimal number between 0-100 followed by a "\%" sign \newline
\noindent Effect\newline
\indent-Possible values:\newline
\indent\indent Any whole/decimal number between 0-100 followed by a "\%" sign \newline
    
\noindent Example: 
\begin{lstlisting}
    Treatment
        Wait
            C:100
        Cost
            C:56
        Effectiveness
            Probability:75%
            Effect:50%
\end{lstlisting}
    
    
\section{Sample File}
\begin{lstlisting}
Simulation
    Name:Sample Configuration
    Output:.txt
    Description:Sample 3-day (plus intro) configuration file  
    Instructions:enabled
    Sound:enabled
Tutorial
    Score:100
Day
    Duration:1:30
Day
    Duration:0:50
    Impairment
        Type:Physical/Shake
        Strength:50%
    Impairment
        Type:Physical/Speed
        Strength:10%
Day
    Duration:2:20
    Impairment
        Type:Visual/Fog
        Strength:75%
    Treatment
        Certainty:80%
        Wait
            C:80
            a:5
            b:6
            c:7
        Cost
            C:98
            a:7
            b:6
            c:5
        Effectiveness
            Probability:50%
            Effect:90%
\end{lstlisting}    

\end{document}
