\documentclass{article}
\usepackage[utf8]{inputenc}

\usepackage{hyperref}
\hypersetup{
    colorlinks=true,
    linkcolor=blue,
    filecolor=magenta,      
    urlcolor=blue,
}

\title{Water Simulation: Application Breakdown}
\author{Aaska Shah\\Kerala Brendon\\Nolan Slade\\Vyome Kishore}
\date{March 2019}

\begin{document}

\maketitle

%% What we could do is include hyperlinks to the classes we mention on GitHub so they
%% could immediately jump to where we're talking about / make the changes they want to

\section*{Overview}

%%%%%%%%%%%%%%%%%%%%%%%%%%%%
\section*{Scene Components}
\subsection*{VR Equipment Interface} % Headset, controllers -> mapping to unity world (virtual hands, etc)

\subsection*{Core Task Functionality} % Bucket, tap, flow manager, destination, limiters, drainage...

\subsection*{Tutorials \& Instructions} % Triggers/markers, object destruction/moving through steps

\subsection*{Receiving Treatment} % Medication station, pills/pedestals/UI, how they are used

\subsection*{Other Features} % Curtains for motion sickness, door for claustrophobia, audio,....


%%%%%%%%%%%%%%%%%%%%%%%%%%%%
\section*{Script Components}
\subsection*{Interactivity} % Virtual hands, camera behaviour, picking up things (vyome's stuff)
\href{https://bit.ly/2Th9Ey3}{\textbf{CameraBehaviour.cs}}: Locates the physical headset object (inside SteamVR CameraRig), then, on every frame, scales its transform positions by the Unity-Vive scale constant (SimManager.UNITY\_VIVE\_SCALE) to accurately map the headset's position to the virtual environment. \newline \newline
\href{https://bit.ly/2U3xT7n}{\textbf{HandTracker.cs}}: This script is attached to each virtual hand game object within the scene. Its job is to take the corresponding physical controller's transform (inside SteamVR CameraRig), then, as in CameraBehaviour, scale the transform values so that virtual controller is placed properly within the scene. Additionally, once the virtual controller's transform has been determined on every frame, the script modifies the transform using user-defined floating point rotation and translation values so that the models of the hands appear to be in a natural position.\newline In addition to positioning a virtual hand, HandTracker also handles haptic feedback for its respective controller, according to the strength of an active impairment. The intensity of the haptic feedback is directly proportional to the strength of such an impairment.

\subsection*{Immersion} % Audio, haptic feedback, etc

\subsection*{Experiment Setup} % Configuration classes / days / SimManager -> state tracking, etc , participant data

\subsection{Task Design}
\subsubsection*{Task Framework} % how the participant can actually do the task. i.e. flow manager, drainage, etc.
\subsubsection*{Impairment} % how we actually coded each impairment, how they get changed per day, etc (sim man)
\subsubsection*{Offering Treatment} % how do we present info / code costs, etc etc / determine which panels to show
\subsubsection*{Receiving Treatment} % how do they trigger an obtain attempt, what happens on a successful attempt
\subsubsection*{Instructions} % Tutorial framework, how it's coded -> beginning + mid-way via limbo / treatments

\subsection*{Custom Configuration} % Aaska's stuff -> using the config file/parsing

\end{document}
