\documentclass[22pt]{beamer}
\usepackage[orientation=portrait, size=custom, width=91.44, height=91.44,scale=1.2]{beamerposter} % 36in*2.5 = 90cm
\usepackage[absolute,overlay]{textpos}
\usepackage{bookmark} %pdflatex says to use this to avoid errors...
\usepackage{graphicx} %for including images
\graphicspath{{figs/}} %location of images
\usepackage{wrapfig} %wrap text around the images
\usepackage{listingsutf8}    %package for code environment; use this instead of verbatim to get automatic line break; use this instead of listings to get (•)
\usepackage{amsmath}
\usepackage{gensymb}
\usepackage[export]{adjustbox}
\usepackage[skins,theorems]{tcolorbox}
\usepackage{tikz}
\newcommand*\circled[1]{\tikz[baseline=(char.base)]{
            \node[shape=circle,draw,inner sep=2pt] (char) {#1};}}
\usepackage{array}
\usepackage{booktabs,adjustbox}
\usepackage{subcaption} 

%\mode<presentation>
%this doesn't seem to make any difference; leave for now for trying out
\usetheme{Berlin}
\definecolor{MacBlue}{rgb}{0.10196,0.22353,0.53725}
\definecolor{MacMaroon} {rgb}{0.47843, 0, 0.23137}
\definecolor{MacMaroon2} {rgb}{0.47451, 0, 0}
\definecolor{MacGray}{rgb}{0.50196,0.49804,0.51765}
\definecolor{MacMaroon3}{rgb}{00.47,0.2,0.31}
\definecolor{MacGold}{rgb}{1, 0.75,0.35}
\usecolortheme[named=MacMaroon2]{structure}
\setbeamertemplate{caption}[numbered]
\setbeamertemplate{navigation symbols}{}

\title{Virtual Reality in Experimental Economics}
\subtitle{}  %probably want a better subtitle
  \author[Shah, Slade, Kishore, Brendon \& Carette]{Aaska Shah, Nolan Slade, Vyome Kishore, Kerala Brendon, supervised by Dr.~Jacques Carette$^\dagger$ \vspace{0.3cm} \newline \small \{vermaa5, yazdinip, irfany1, schankuc, anandc\}@mcmaster.ca}
  \institute[McMaster University]{$^\dagger$Department of Computing and Software, McMaster University

1280 Main St. W, Hamilton, Ontario, Canada L8S 4L8}
  \date{December 5th, 2018}

\begin{document}
%compile with pdflatex

%there is only one frame, because there is only one page; yeah, it's a poster
%textblock and block seem to work nicely to organize layout
\begin{frame}[fragile]

\begin{textblock}{2}(0.7,1)
\includegraphics[height=8.5cm]{englogo.png} % We can use CAS logo as well? 
\end{textblock}

\begin{textblock}{8}(4,1)
\titlepage
\end{textblock}

\begin{textblock}{7.25}(0.5,3.1)

%this needs help
\begin{block}{Introduction}
For our final year computer science capstone project at McMaster University, we are partnering with the McMaster Decision Science Laboratory to create 2 virtual reality-based economics research simulations. Each simulation is designed
around the user repetitively completing a basic task with the goal of maximizing
earnings that are awarded on every completed iteration of the task. The
simulations are structured such that their lifetimes are separated into discrete
time periods, days, in which the user’s ability to perform the repetitive task is
hindered by an impairment(s). The user will be given the option to reduce the
effects of such an impairment by receiving a treatment - such treatments can be
paid for, or received for free after waiting a predetermined duration. Each simulation will feature both a virtual environment designed within Unity
and HTC Vive support to offer a truly immersive simulation experience to the
participant.

\end{block}

\begin{block}{The Original Simulation}
\begin{itemize}
\item Crate carrying exercise (Figure \ref{fig:orig} (a))
\item Large environment that does not scale to experiment room (Figure \ref{fig:orig} (b))
\item No extensive customization of configuration variables.
\end{itemize}
%\includegraphics[height=12cm, width=15cm]{experiment.png}. %diagram only/control
%\includegraphics[height=12cm, width=15cm]{experiment2.png} %second group
%\includegraphics[height=12cm, width=15cm]{3rd image?}

\begin{figure}
  %
  \begin{subfigure}[t]{0.30\textwidth}
    \includegraphics[height=13cm]{ExperimentSeparate.png}
    \caption{Diagram, information separate}				%confirm the cases 
    \label{a} 
  \end{subfigure}
    \begin{subfigure}[t]{0.41\textwidth}
    \includegraphics[height=13cm]{experiment.png}
    \caption{Diagram with integrated captions}				%confirm the cases 
    \label{a} 
  \end{subfigure}    
  \caption{Original cognition figure from \cite{cerpa1996some}. a) shows the diagram with separate captioning, presented to group A; b) with integrated captioning was presented to group B\label{fig:orig}}
\end{figure}
{\tiny $^1$ There was a third case in the original study which is not relevant to the current study.}
\end{block}

\begin{block}{Our Simulations}
\begin{itemize}
\item Allow for complete customization of the experiment domain through a  comprehensive configuration file
\item Uses a SQL database for improved data storage
\item Scaled to the size of Vive equiped the testing room
\end{itemize}
\begin{figure}
  \begin{subfigure}[b]{0.34\textwidth}
    \includegraphics[width=\textwidth]{Demo_1.png}
    \caption{}
    \label{fig:1}
  \end{subfigure}
  \begin{subfigure}[b]{0.339\textwidth}
    \includegraphics[width=\textwidth]{Demo_12.png}
    \caption{}
    \label{fig:2}
  \end{subfigure}

      \caption{An example teaching the code for creating transparent shapes, corresponding to \#4 on Table \ref{tab:summ}. Parts a), b) and c) gradually introduce circles of different transparencies alongside the code used to generate the output.}
\end{figure}
\end{block}


\end{textblock}



\begin{textblock}{7.25}(8.25,3.1)

\begin{block}{Simulation 1}
In this simulation, the participant will be required to repeatedly transport a
volume of liquid between a source and destination using a single hand-carried
vessel. Their goal will be to maximize the total volume of liquid that successfully
reaches the destination.

\begin{figure}
  \begin{subfigure}[b]{0.34\textwidth}
    \includegraphics[width=\textwidth]{Demo_1.png}
    \caption{}
    \label{fig:1}
  \end{subfigure}
  \begin{subfigure}[b]{0.339\textwidth}
    \includegraphics[width=\textwidth]{Demo_12.png}
    \caption{}
    \label{fig:2}
  \end{subfigure}

      \caption{An example teaching the code for creating transparent shapes, corresponding to \#4 on Table \ref{tab:summ}. Parts a), b) and c) gradually introduce circles of different transparencies alongside the code used to generate the output.}
\end{figure}

%%Have to add these
 %Figure #. Taken from the night demo shows the application of transparency in animation



\begin{comment}
\includegraphics[height = 9.5cm, width = 12cm]{Demo_1.png}. 
\includegraphics[height = 9.5cm, width = 12cm]{Demo1mid.png}. 
\includegraphics[height = 9.5cm, width = 12cm]{Demo_12.png}\newline. % Figure #. Taken from the Transparecny cool demo, each picture represents the animation at 3 differnet stages highlighting the different between the value given to the fucntion
\end{comment}



\end{block}
\begin{block}{Simulation 2}
The second simulation will provide the participant with a set of three-dimensional
shapes, and task them to sort them into separate containers by passing them
through a filter that only permits one particular shape. The aim of the participant
will be to maximize the total number of shapes successfully sorted into
their respective containers.
\begin{figure}
  \begin{subfigure}[b]{0.40\textwidth}
    \includegraphics[height=11cm]{shape_sorting_environment.png}
        \caption{Layout of simulation environment.}
  \end{subfigure}
  %
  \begin{subfigure}[b]{0.40\textwidth}
    \includegraphics[height=11cm]{shape_shorting_filter.png}
        \caption{Shape sorting filter.}
    \end{subfigure}
 \end{figure}
\end{block}


\begin{block}{Summary of Configuration Variables}
Experiment varibles to be set in the configuration file.
\begin{itemize}
\item Money acquired per accomplished task
\item Cost of treatment
\item Waiting time for option to purchase treatment
\item Waiting time for option to receive treatment free of charge
\item Waiting time after treatment before health restored
\item Number of "days" the experiment runs
\item "Days" the participant is impaired
\item "Days" free treatment is offered
\item "Days" payed treatment is offered
\item Health level on impaired days
\item Health gained from treatment
\item Impairment type
\item Intensity of impairment
\end{itemize}




\begin{comment}
\begin{center}
\begin{table}
\caption{Descriptions of selected demos\label{tab:summ}}
\begin{tabular}{ | m{1.2em} | m{7em} | m{25cm}| } 
\hline
\# & Name & Purpose  \\ 
\hline
1 & Ngon & Introduction to polygons with equal-length sides\\ 
\hline
2 & Scale & Introduction to scale function\\
\hline
3 & Outline & Styles of Outlines \\
\hline
4 & Transparency & Controlling the transparency of shapes\\
\hline
5 & Cookies & Introduction to move \\  
\hline
6 & Rotation & Rotating objects using degrees\\
\hline
7 & Pile Up & Explaining how order code affects layering of shapes\\ 
\hline
8 & Rainbow & Application of order\\ 
\hline
9 & Night & Application of draw order and transparency \\ 
\hline
10 & Candles & Application of scale and colours  \\ 
\hline
11 &  Colourful Flower & Application of rotation \\ 
 \hline
 12 & Apple & Introduction to animation in rotation \\ 
 \hline
13 &  Car & Creating simple scenes \\ 
 \hline
14 &  Door & Inverting shape using negative scale \\  
\hline
15 &  PacMan & Animating movement \\ 
\hline
16 &  Raindrop & Introduction to creating objects \\  
 \hline


\end{tabular}
\end{table}
\end{center}
\end{comment}
\end{block}

\begin{block}{Conclusions \& Future Work}
Using the existing experiment and while consulting with the McMaster Decision Science Laboratory we have developed a plan and have begun implementing the two simulations as described here. These simulations will allow the laboratory to run dynamic a unique experiments using either of the simulations with specific configurations. The simulations will be tailored to the test room so the overall experience is as realistic as possible.
\end{block}

\begin{block}{Acknowledgements}

\end{block}

\begin{block}{References}
\setbeamertemplate{bibliography item}{\insertbiblabel}
\bibliographystyle{ieeetr}
{\scriptsize
\bibliography{bib}}
\end{block}

\begin{comment}
%these aren't in any particular style, it's just the basic idea
\begin{block}{References}
\setbeamertemplate{bibliography item}{\insertbiblabel}
\bibliographystyle{ieeetr}
{\scriptsize
\bibliography{bib}}
\end{block}
\vspace{-1.8mm}
%will need some more graphics to thank the various people
\end{comment}
\end{textblock}


\end{frame}
\end{document}
