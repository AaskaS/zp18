\documentclass[22pt]{beamer}
\usepackage[orientation=portrait, size=custom, width=91.44, height=91.44,scale=1.2]{beamerposter} % 36in*2.5 = 90cm
\usepackage[absolute,overlay]{textpos}
\usepackage{bookmark} %pdflatex says to use this to avoid errors...
\usepackage{graphicx} %for including images
\graphicspath{{figs/}} %location of images
\usepackage{wrapfig} %wrap text around the images
\usepackage{listingsutf8}    %package for code environment; use this instead of verbatim to get automatic line break; use this instead of listings to get (•)
\usepackage{amsmath}
\usepackage{gensymb}
\usepackage[export]{adjustbox}
\usepackage[skins,theorems]{tcolorbox}
\usepackage{tikz}
\newcommand*\circled[1]{\tikz[baseline=(char.base)]{
            \node[shape=circle,draw,inner sep=2pt] (char) {#1};}}
\usepackage{array}
\usepackage{booktabs,adjustbox}
\usepackage{subcaption} 

%\mode<presentation>
%this doesn't seem to make any difference; leave for now for trying out
\usetheme{Berlin}
\definecolor{MacBlue}{rgb}{0.10196,0.22353,0.53725}
\definecolor{MacMaroon} {rgb}{0.47843, 0, 0.23137}
\definecolor{MacMaroon2} {rgb}{0.47451, 0, 0}
\definecolor{MacGray}{rgb}{0.50196,0.49804,0.51765}
\definecolor{MacMaroon3}{rgb}{00.47,0.2,0.31}
\definecolor{MacGold}{rgb}{1, 0.75,0.35}
\usecolortheme[named=MacMaroon2]{structure}
\setbeamertemplate{caption}[numbered]
\setbeamertemplate{navigation symbols}{}

\title{Virtual Reality in Experimental Economics}
\subtitle{Capstone Project: Computer Science} 
  \author[Shah, Brendon, Slade, Kishore \& Carette]{Aaska Shah, Kerala Brendon, Nolan Slade, Vyome Kishore, supervised by Dr.~Jacques Carette$^\dagger$ \vspace{0.3cm} \newline \small \{shaha8, brendokh, sladenj, kishov, anandc\}@mcmaster.ca, stephanie.thomas1@curtin.edu.au and Team}
  \institute[McMaster University]{$^\dagger$Department of Computing and Software, McMaster University

1280 Main St. W, Hamilton, Ontario, Canada L8S 4L8}
  \date{December 5th, 2018}

\begin{document}
%compile with pdflatex

%there is only one frame, because there is only one page; yeah, it's a poster
%textblock and block seem to work nicely to organize layout
\begin{frame}[fragile]

\begin{textblock}{2}(0.7,1)
\includegraphics[height=8.5cm]{englogo.png} % We can use CAS logo as well? 
\end{textblock}

\begin{textblock}{8}(4,1)
\titlepage
\end{textblock}

\begin{textblock}{7.25}(0.5,3.1)

%this needs help
\begin{block}{Project Overview}\newline
\begin{itemize}
\item \textbf{Goal}: develop immersive virtual reality environments, as an extension of previous work done by the McMaster Decision Science Laboratory, to carry out experimental economics research simulations to gain insight into how participants make health-related decisions. 
\item Simulations are structured around the participant maximizing earnings that are awarded on every completed iteration of some basic task.
\item The simulations\textsc{\char13} lifetimes are separated into discrete time periods, days, in which the participant\textsc{\char13}s
ability to perform the repetitive task can be hindered by an impairment(s).
\item Participants can choose to alleviate the symptoms of such impairments by receiving treatment; but, it comes at a cost. The question is when will the participant resort to paying the necessary cost, or will they do so at all? 
\item A suite of simulation environments will be designed within the Unity game engine; participants will interact with each simulation through an HTC Vive. 
\end{itemize}
\end{block}

\begin{block}{Existing Simulation}\newline
\begin{itemize}
\item The participant's task is to move crates from a pile to the target block.
\item The environment is large and does not scale to the Vive-equipped testing room (Figure \ref{fig:crate}).
\item No extensive customization of configuration variables.
\end{itemize}
\newline
\begin{figure}
  \includegraphics[height=10cm]{CrateOriginal.PNG}
  \caption{Experiment room}
\label{fig:crate}
\end{figure}
\end{block}


\begin{block}{Our Simulations}\newline
\begin{itemize}
\item Each simulation features its own basic task for the experiment participant to complete.
\item Their respective virtual environments and experiment structures will be highly customizable through the usage of a proprietary configuration filetype. 
\item Key metrics will be tracked throughout the simulations\textsc{\char13} lifetimes and persisted into a database for analysis.
\item Each environment is carefully designed with consideration towards the physical constraints of the Vive-equipped testing room (Figure \ref{fig:room}).
\end{itemize}
\newline
\begin{figure}
  \includegraphics[height=10cm]{NolanVRRoom.png}
  \caption{Experiment room}
\label{fig:room}
\end{figure}
\end{block}


\begin{block}{Summary of Configuration Variables}\newline
Experiment variables to be set in the aforementioned configuration file. These variables include:
\begin{itemize}
\item Reward obtained by the participant for every completed task iteration.
\item Impairment types and their respective intensities.
\item Treatment methods, costs, and effectiveness.
\item Per-day settings such as duration and active impairment(s).
\end{itemize}
\end{block}



%\begin{block}{Conclusions \& Future Work}
%\begin{itemize}
%\item Using the existing experiment and consulting with the McMaster Decision Science Laboratory, we have developed a plan to implement the two simulations as described.
%\item These simulations will allow the laboratory to run unique experiments using either of the simulations with specific configurations while collecting data in a SQL database.
%\item The simulations will be tailored to the Vive equipped test room so the overall experience is as realistic as possible.
%\end{itemize}
%\end{block}

\end{textblock}




\begin{textblock}{7.25}(8.25,3.1)

\begin{block}{Simulation One}\newline
\begin{itemize}
\item The participant is required to repeatedly transport a
volume of liquid between a source and destination using a single hand-carried
vessel.
\item Their goal is to maximize the total volume of liquid that successfully
reaches the destination.
\end{itemize}

\begin{figure}
  \centering
  \includegraphics[height=10cm]{SimulationOne.png}
  \caption{First simulation environment}
\label{fig:simOne}
\end{figure}

%%Have to add these
 %Figure #. Taken from the night demo shows the application of transparency in animation
\end{block}



\begin{block}{Simulation Two}\newline
\begin{itemize}
\item The participant is required to sort a set of three-dimensional shapes into separate containers by passing them through a filter that only permits one particular shape.
\item The goal is to maximize the total number of shapes successfully sorted into their respective containers.
\end{itemize}
\begin{figure}
  \begin{subfigure}[b]{0.40\textwidth}
    \includegraphics[height=11cm]{shape_sorting_environment.png}
        \caption{Layout of simulation environment.}
  \end{subfigure}
 %
  \begin{subfigure}[b]{0.40\textwidth}
    \includegraphics[height=11cm]{shape_shorting_filter.png}
        \caption{Shape sorting filter.}
    \end{subfigure}
 \end{figure}
\end{block}


\begin{block}{Sound Interesting?}\newline
Send us what you think of our project, and tell us if you\textsc{\char13}d like to participate in experimental trials.

\newline
\newline
Email any of us: \newline
\newline
Aaska: shaha8@mcmaster.ca\newline
Nolan: sladenj@mcmaster.ca\newline
Kerala: brendokh@mcmaster.ca\newline
Vyome: kishov@mcmaster.ca\newline
\newline
And CC our coordinator:\newline
Dr. Christopher Anand: anandc@mcmaster.ca
\newline
\newline
Experimental trials will likely be scheduled for February 2019.
\end{block}


\begin{block}{Acknowledgements}\newline
We wish to extend our thanks to the faculty and staff at the McMaster Decision Science Laboratory for their guidance as well as their accomodation in allowing us access to lab resources as we move forward with this exciting project. Notably:\newline
\newline
Stephanie Thomas: \textit{School of Economics and Finance, Curtin University}\newline
David Cameron: \textit{Manager, McMaster Decision Science Laboratory}\newline
Neil Buckley: \textit{Associate Professor, York University}\newline
Courtney Sheppard: \textit{IT Advisor, McMaster Decision Science Laboratory}
\end{block}

% \begin{block}{References}
% \setbeamertemplate{bibliography item}{\insertbiblabel}
% \bibliographystyle{ieeetr}
% {\scriptsize
% \bibliography{bib}}
% \end{block}

\begin{comment}
%these aren't in any particular style, it's just the basic idea
\begin{block}{References}
\setbeamertemplate{bibliography item}{\insertbiblabel}
\bibliographystyle{ieeetr}
{\scriptsize
\bibliography{bib}}
\end{block}
\vspace{-1.8mm}
%will need some more graphics to thank the various people
\end{comment}
\end{textblock}


\end{frame}
\end{document}
