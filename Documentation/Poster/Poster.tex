\documentclass[22pt]{beamer}
\usepackage[orientation=portrait, size=custom, width=91.44, height=91.44,scale=1.2]{beamerposter} % 36in*2.5 = 90cm
\usepackage[absolute,overlay]{textpos}
\usepackage{bookmark} %pdflatex says to use this to avoid errors...
\usepackage{graphicx} %for including images
\graphicspath{{figs/}} %location of images
\usepackage{wrapfig} %wrap text around the images
\usepackage{listingsutf8}    %package for code environment; use this instead of verbatim to get automatic line break; use this instead of listings to get (•)
\usepackage{amsmath}
\usepackage{gensymb}
\usepackage[export]{adjustbox}
\usepackage[skins,theorems]{tcolorbox}
\usepackage{tikz}
\newcommand*\circled[1]{\tikz[baseline=(char.base)]{
            \node[shape=circle,draw,inner sep=2pt] (char) {#1};}}
\usepackage{array}
\usepackage{booktabs,adjustbox}
\usepackage{subcaption} 

%\mode<presentation>
%this doesn't seem to make any difference; leave for now for trying out
\usetheme{Berlin}
\definecolor{MacBlue}{rgb}{0.10196,0.22353,0.53725}
\definecolor{MacMaroon} {rgb}{0.47843, 0, 0.23137}
\definecolor{MacMaroon2} {rgb}{0.47451, 0, 0}
\definecolor{MacGray}{rgb}{0.50196,0.49804,0.51765}
\definecolor{MacMaroon3}{rgb}{00.47,0.2,0.31}
\definecolor{MacGold}{rgb}{1, 0.75,0.35}
\usecolortheme[named=MacMaroon2]{structure}
\setbeamertemplate{caption}[numbered]
\setbeamertemplate{navigation symbols}{}

\title{Virtual Reality in Experimental Economics}
\subtitle{Computer Science Capstone Project} 
  \author[Shah, Brendon, Slade, Kishore \& Carette]{Aaska Shah, Kerala Brendon, Nolan Slade, Vyome Kishore, supervised by Dr.~Jacques Carette$^\dagger$ \vspace{0.3cm} \newline \small \{shaha8, brendokh, sladenj, kishov, anandc\}@mcmaster.ca, stephanie.thomas1@curtin.edu.au and Team}
  \institute[McMaster University]{$^\dagger$Department of Computing and Software, McMaster University

1280 Main St. W, Hamilton, Ontario, Canada L8S 4L8}
  \date{December 5th, 2018}

\begin{document}
%compile with pdflatex

%there is only one frame, because there is only one page; yeah, it's a poster
%textblock and block seem to work nicely to organize layout
\begin{frame}[fragile]

\begin{textblock}{2}(0.7,1)
\includegraphics[height=8.5cm]{englogo.png} % We can use CAS logo as well? 
\end{textblock}

\begin{textblock}{8}(4,1)
\titlepage
\end{textblock}

\begin{textblock}{7.25}(0.5,3.1)

%this needs help
\begin{block}{Introduction}
\begin{itemize}
\item Creating two virtual reality-based economics research simulations in collaboration with the McMaster Decision Science Laboratory.
\item The participant repetitively completes a basic task with the goal of maximizing
earnings that are awarded on every completed iteration of the task.
\item The simulation's lifetime is separated into discrete time periods, days, in which the participant's ability to perform the repetitive task can be hindered by an impairment(s).
\item The participant can reduce the
effects of such an impairment by receiving a treatment, which can be paid for or received for free after waiting for a predetermined duration
\item Each simulation will feature both a virtual environment designed within Unity
and HTC Vive support to offer a truly immersive simulation experience to the
participant.
\end{itemize}

\end{block}


\begin{block}{Our Simulations}
\begin{itemize}
\item Two new participant tasks to use in experiments.
\item Allow for complete customization of the experiment domain through a  comprehensive configuration file.
\item Use a SQL database for improved data storage.
\item Scaled to the size of the Vive equipped testing room (Figure \ref{fig:room}).
\end{itemize}
\begin{figure}
    \includegraphics[height=13cm]{NolanVRRoom.png}
  \caption{VR Experiment Room}
\label{fig:room}
\end{figure}
\end{block}

\begin{block}{Simulation 2}
\begin{itemize}
\item The participant is required to sort a set of three-dimensional
shapes into separate containers by passing them
through a filter that only permits one particular shape.
\item The goal is to maximize the total number of shapes successfully sorted into
their respective containers.
\end{itemize}
\begin{figure}
  \begin{subfigure}[b]{0.40\textwidth}
    \includegraphics[height=11cm]{shape_sorting_environment.png}
        \caption{Layout of simulation environment.}
  \end{subfigure}
 %
  \begin{subfigure}[b]{0.40\textwidth}
    \includegraphics[height=11cm]{shape_shorting_filter.png}
        \caption{Shape sorting filter.}
    \end{subfigure}
 \end{figure}
\end{block}


\begin{block}{Conclusions \& Future Work}
\begin{itemize}
\item Using the existing experiment and consulting with the McMaster Decision Science Laboratory, we have developed a plan to implement the two simulations as described.
\item These simulations will allow the laboratory to run unique experiments using either of the simulations with specific configurations while collecting data in a SQL database.
\item The simulations will be tailored to the Vive equipped test room so the overall experience is as realistic as possible.
\end{itemize}
\end{block}

\end{textblock}




\begin{textblock}{7.25}(8.25,3.1)

\begin{block}{The Original Simulation}
\begin{itemize}
\item User task is to carry crates to the designated point.
\item Large environment that does not scale to Vive equipped testing room (Figure \ref{fig:orig}).
\item No extensive customization of configuration variables.
\end{itemize}
\begin{figure}
    \includegraphics[height=13cm]{CrateOriginal.PNG}
  \caption{Original Experiment Simulation}
\label{fig:orig}
\end{figure}
\end{block}

\begin{block}{Simulation 1}
\begin{itemize}
\item The participant is required to repeatedly transport a
volume of liquid between a source and destination using a single hand-carried
vessel.
\item Their goal is to maximize the total volume of liquid that successfully
reaches the destination.
\end{itemize}

\begin{figure}
\begin{figure}
    \includegraphics[height=13cm]{SimulationOne.png}
  \caption{Simulation One Environment}
\label{fig:room}
\end{figure}
\end{figure}

%%Have to add these
 %Figure #. Taken from the night demo shows the application of transparency in animation



\end{block}

\begin{block}{Summary of Configuration Variables}
Experiment variables to be set in the configuration file.
\begin{itemize}
\item Money acquired per accomplished task
\item Cost of treatment
\item Waiting duration for option to purchase treatment, or receive free of charge
\item Waiting time between treatment and health restored
\item "Day" configurations, such as the days the participants are impaired
\item Health level on impaired days and the health gained from a treatment
\item Impairment type and the intensity of impairment
\end{itemize}
\end{block}




\begin{block}{Acknowledgements}
Thank you to the faculty and staff at the McMaster Decision Science Laboratory for you guidance and use of lab space.
\end{block}

\begin{block}{References}
\setbeamertemplate{bibliography item}{\insertbiblabel}
\bibliographystyle{ieeetr}
{\scriptsize
\bibliography{bib}}
\end{block}


\begin{comment}
%these aren't in any particular style, it's just the basic idea
\begin{block}{References}
\setbeamertemplate{bibliography item}{\insertbiblabel}
\bibliographystyle{ieeetr}
{\scriptsize
\bibliography{bib}}
\end{block}
\vspace{-1.8mm}
%will need some more graphics to thank the various people
\end{comment}
\end{textblock}


\end{frame}
\end{document}
