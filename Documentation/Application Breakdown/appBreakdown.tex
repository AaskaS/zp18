\documentclass{article}
\usepackage[utf8]{inputenc}

\usepackage{hyperref}
\hypersetup{
    colorlinks=true,
    linkcolor=blue,
    filecolor=magenta,      
    urlcolor=blue,
}

\title{Water Simulation: Application Breakdown}
\author{Aaska Shah\\Kerala Brendon\\Nolan Slade\\Vyome Kishore}
\date{March 2019}

\begin{document}

\maketitle

%% What we could do is include hyperlinks to the classes we mention on GitHub so they
%% could immediately jump to where we're talking about / make the changes they want to

\section*{Overview}

%%%%%%%%%%%%%%%%%%%%%%%%%%%%
\section*{Scene Components}
\subsection*{VR Equipment Interface} % Headset, controllers -> mapping to unity world (virtual hands, etc



\subsection*{Core Task Functionality} % Bucket, tap, flow manager, destination, limiters, drainage...

\textbf{Container\_Base}: The container game object represents the bucket the subject carriers to move water from one point to another. It contains 5 cube objects that make up the walls of the bucket (\textit{Container\_Wall\_Front}, \textit{Container\_Wall\_Back}, \textit{Container\_Wall\_Right}, \textit{Container\_Wall\_Left}). It also contains an invisible collider (\textit{Water\_Droplet\_Counter}) to count water droplets that enter and exit the bucket, using the \textit{WaterDropletCounter} script.\newline \newline
%%% What is the AttachTransformRight and AttachTransformLeft?
\textbf{SimManager}: A game object to hold the \textit{SimManager} script.\newline \newline
\textbf{PillManager}: A game object to hold the \textit{PillManager} script.\newline \newline
\textbf{FlowManager}: A game object to hold the \textit{FlowManager} script.\newline \newline
\textbf{FlowLimiter}: An invisible cube object with a collider to determine if the headset is close enough to the source for the flow to be activated. Contains the \textit{FlowLimiter} script.\newline \newline
\textbf{DestinationLimiter}: An invisible cube object with a collider to determine if the headset is close enough to the destination to accept droplets. Contains the \textit{DestinationLimiter} script.\newline \newline
\textbf{DayZeroSpeedCounter}: Currently not in use, to be used in future implementation of speed penalty to measure the subject's average speed during day 0 tutorial.\newline \newline
\textbf{Drainage}: A cube object placed at the bottom of the source sink to hold the \textit{DrainageBehaviour} script which controls disappearing droplets.
\textbf{TargetDrainage}: A cube object placed at the bottom of the destination sink to hold the \textit{DrainageBehaviour} script which controls disappearing droplets and counting delivered droplets.\newline \newline
\textbf{CameraRig/Camera/Canvas/FogImpairmentPanel}: This canvas is attached to the camera is used to make the subject's vision appear foggy. The opacity is determined by the intensity of the fog impairment.\newline \newline


\subsection*{Tutorials \& Instructions} % Triggers/markers, object destruction/moving through steps
\textbf{InstructionMan}: The game object contains the \textit{InstructionManager} script.\newline \newline
\textbf{BucketMarkerTrigger}: This game object is an invisible cube with a collider to determine if the headset has approached the bucket marker.\newline \newline
\textbf{BucketPickedUpTrigger}: This game object is an invisible cube with a collider to determine if the headset has approached the bucket marker.\newline \newline
\textbf{BucketMarker}: This game object is a red exclamation point used to notify the subject of the bucket's location.\newline \newline
\textbf{BucketPickedUpTrigger}: %what does this do?
\textbf{FarSinkMarker}: This game object is a red exclamation point used to notify the subject of the destination sink bucket.\newline \newline
\textbf{CameraRig/Camera/Canvas/InstructionPanel}: This canvas is attached to the camera so that it appears in the view at all times and moves with the headset. It contains \textit{InstructionTxt} which will display instructions to the subject during the tutorial. \newline \newline
\textbf{CameraRig/Camera/Canvas/TransitionPanel}: This canvas is attached to the camera so that it appears in the view at all times and moves with the headset. It contains \textit{TransitionCountdown} which will display the number of seconds until the new day begins. \newline \newline

\subsection*{Receiving Treatment} % Medication station, pills/pedestals/UI, how they are used
\textbf{TreatmentInformationCanvas}: This canvas contains a panel that is placed above the treatment table. It contains \textit{WaitRemainingPanel} for displaying the wait time when only wait time is available \textit{WaitPanel} for displaying the wait time for when both pay and wait is available. \textit{PayPanel} is for displaying the treatment cost. \textit{TreatmentInformation} is for displaying the title. \newline \newline
\textbf{PayPills}: This game object is a prefab pill bottle. The \textit{Throwable}, \textit{Interactable} and \textit{VelocityEstimator} scripts allows it to be picked up with the controller. \newline \newline
\textbf{PayPedestal}: This game object is a red cylinder object for the \textit{PayPills} object to be placed on. \newline \newline
\textbf{WaitPedestal}: This game object is a blue cylinder object for the \textit{WaitPills} object to be placed on when pay is also an option. \newline \newline
\textbf{WaitPlatform}: This game object is a blue cube object for the \textit{WaitPills} object to be placed on when it is the only option. \newline \newline
\textbf{MedicalCross}: This game object is two red cube objects placed to appear like a cross to indicate the medical station. \newline \newline

\subsection*{Other Features} % Curtains for motion sickness, door for claustrophobia, audio,....
\textbf{Source\_Dispense\_Pipe}: This is a pipe prefab placed above the sink that makes it appear that the water droplets are being dispensed from the pipe. No functionality is attached to this game object.\newline \newline
\textbf{SourceTubPipe}: This is a pipe prefab placed below the source sink.\newline \newline
\textbf{DestinationTubPipe}: This is a pipe prefab placed below the destination sink.\newline \newline
\textbf{Window\_Frames}: Set of cube objects that make up the frames for the windows, along with two for the window glass.\newline \newline
\textbf{Door}: A prefab door and two paintings on the left wall used when the subject suffers from motion sickness.\newline \newline
\textbf{AudioManager}: An audio object that contains audio clips for dispensing the water, pouring the water, countdown sound, grabbing pills sound and completion jingle. The \textit{AudioManager} script to determine when to play the audio clips.\newline \newline
\textbf{Source\_Tub}: A prefab tub under where the droplets are dispensed.\newline \newline
\textbf{Destination\_Tub}: A prefab tub for the subject to pour droplets into.\newline \newline
\textbf{Carpet}: A prefab carpet for the floor.\newline \newline
\textbf{Trees}: Prefab tress for the outdoor scene. \newline \newline
\textbf{Ceiling}: The ceiling for the house, made up of square \textit{PTK\_Wall\_Small\_1}. \newline \newline
\textbf{Curtain\_Left}: A curtain to be placed over the left window if the subject gets motion sickness. \newline \newline
\textbf{Curtain\_Right}: A curtain to be placed over the right window if the subject gets motion sickness. \newline \newline
\textbf{Wall\_Sink}: The wall on the sink side \newline \newline
\textbf{Wall\_Source}: A curtain to be placed over the right window if the subject gets motion sickness. \newline \newline
\textbf{Wall\_Left}: A curtain to be placed over the right window if the subject gets motion sickness. \newline \newline
\textbf{Wall\_Right}: A curtain to be placed over the right window if the subject gets motion sickness. \newline \newline
\textbf{Column\_Left\_Rear}: A curtain to be placed over the right window if the subject gets motion sickness. \newline \newline
\textbf{Column\_Left\_Front}: A curtain to be placed over the right window if the subject gets motion sickness. \newline \newline

%%%%%%%%%%%%%%%%%%%%%%%%%%%%    \href{}{\textbf{.cs}}:  \newline \newline
\section*{Script Components}
\subsection*{Interactivity} % Virtual hands, camera behaviour, picking up things (vyome's stuff)
\href{https://bit.ly/2JH9ws0}{\textbf{CameraBehaviour.cs}}: Locates the physical headset object (inside SteamVR CameraRig), then, on every frame, scales its transform positions by the Unity-Vive scale constant (SimManager.UNITY\_VIVE\_SCALE) to accurately map the headset's position to the virtual environment. \newline \newline
\href{https://bit.ly/2U3xT7n}{\textbf{HandTracker.cs}}: This script is attached to each virtual hand game object within the scene. Its job is to take the corresponding physical controller's transform (inside SteamVR CameraRig), then, as in CameraBehaviour, scale the transform values so that virtual controller is placed properly within the scene. Additionally, once the virtual controller's transform has been determined on every frame, the script modifies the transform using user-defined floating point rotation and translation values so that the models of the hands appear to be in a natural position.\newline In addition to positioning a virtual hand, HandTracker also handles haptic feedback for its respective controller, according to the strength of an active impairment. The intensity of the haptic feedback is directly proportional to the strength of such an impairment. \newline \newline
\href{}{\textbf{Hand.cs}}: Open source Steam VR hand script to assist with interactability of hands with other assets in the scene. Modified to work with our assets and functionalities.


\subsection*{Immersion} % Audio, haptic feedback, etc
\href{https://bit.ly/2HHzDwj}{\textbf{AudioManager.cs}}: Offers a public method to play an array of sound effects, including water flowing, medicine consumption, day start, day end, simulation end, as well as countdowns. Each one of these sound effects is attached within the editor onto an unassigned public AudioClip variable. Multiple AudioManagers can be placed within the scene in case of potential conflicts; for example, water could flow at the same time a countdown is taking place. The script also includes methods to mute all sounds, as well as stop the current sound. All supported sound effects are defined in the AudioManager.\textit{SoundType} enumerated type. Each AudioManager is designed to be attached to an empty game object within the scene.


\subsection*{Experiment Setup} % Configuration classes / days / SimManager -> state tracking, etc
\href{https://bit.ly/2FvRTWR}{\textbf{DayConfiguration.cs}}: Encapsulates basic information for each day of the simulation. Specifically, their unique number identifiers (int), durations in seconds (float), impairments, treatment options, and optionally, how much to pay the participant for each drop of delivered water (float) on the given day. In addition to two constructors, this script also offers accessors for each one of these key components. \newline \newline
\href{https://bit.ly/2TZaLYj}{\textbf{ConfigParser.cs : \textit{ ()}}}: TODO \newline \newline
\href{https://bit.ly/2UhmSzq}{\textbf{SimManager.cs : \textit{establishSimulationParameters ()}}}: TODO \newline \newline
\href{https://bit.ly/2UhmSzq}{\textbf{SimManager.cs : \textit{getCurrentDayConfiguration ()}}}: TODO \newline \newline
\href{https://bit.ly/2UhmSzq}{\textbf{SimManager.cs : \textit{Update ()}}}: TODO 

\subsection*{Task Design}
\subsubsection*{Task Framework} % how the participant can actually do the task. i.e. flow manager, drainage, etc.
\href{https://bit.ly/2WrbyxT}{\textbf{FlowManager.cs}}: \newline \newline 
\href{https://bit.ly/2YtMEjd}{\textbf{DrainageBehaviour.cs}}: TODO  \newline \newline 
\href{https://bit.ly/2TUX1h4}{\textbf{FlowLimiter.cs}}: Prevents the tap from flowing except when the participant is standing close enough. The main idea behind this script's implementation was to prevent cheating in that players are forced to physically cross the room rather than reach from one side to the other while standing relatively still. The script is meant to be attached to an invisible game object with a collider enabled (the collider must have \textit{isTrigger} set to true). Collisions will then be detected between this invisible game object and the participant's headset; no collisions will be detected with the participant's virtual hands. The script is also designed to prevent the tap from flowing if the participant is waiting for treatment, or if the simulation is not in the \textit{RUNNING} state. \newline \newline
\href{https://bit.ly/2JGhr8M}{\textbf{DestinationLimiter.cs}}: Similar to FlowLimiter, except instead of preventing water from flowing, this script prevents the participant from being payed unless they are standing close enough to the destination sink. This script is also meant to be used on an invisible game object with a collider attached (and \textit{isTrigger} = true). 

\subsubsection*{Impairment} % how we actually coded each impairment, how they get changed per day, etc (sim man)
\href{https://bit.ly/2JFJ9mj}{\textbf{Impairment.cs}}: Encapsulates the two components of an impairment: the type (defined in the Treatment.\textit{ImpairmentType} enum), and strength (float, percentage expressed from 0.0 to 1.0). Offers public methods to retrieve and set each one of these components. \newline \newline
\href{https://bit.ly/2UhmSzq}{\textbf{SimManager.cs : \textit{Update ()}}}: TODO \newline \newline % How/where do we apply them in update
\href{https://bit.ly/2UhmSzq}{\textbf{SimManager.cs : \textit{modifyImpairmentFactors (float)}}}: Iterates over all active impairments for the current day, and decreases their respective strengths by the given factor. For example, if all impairments have a 50\% strength, calling \texit{modifyImpairmentFactors (0.75)} will yield new impairment strengths of 12.5\%; or, specifically, strength = original\_strength * (1 - factor).  \newline \newline
\href{https://bit.ly/2UhmSzq}{\textbf{SimManager.cs : \textit{unapplyImpairments ()}}}: Iterates over all active impairments for the current day (if any), and deactivates them accordingly. Depending on the impairment, the operations to deactive will vary. Functionally, this method is equivalent to calling \textit{modifyImpairmentFactors (1.0)}, that is, remove 100\% of the strength for each active impairment. 

\subsubsection*{Offering Treatment} % how do we present info / code costs, etc etc / determine which panels to show
\href{https://bit.ly/2TwrvAZ}{\textbf{Treatment.cs}}: Encapsulates all components that make up a given day's treatment options. These components include private float members for C, a, b, c of both the day's pay cost function and its wait time cost function (wait\_C, wait\_a, wait\_b, wait\_c and cost\_C, cost\_a, cost\_b, and cost\_c, respectively). If a treatment day only offers the choice of paying, then the wait members are set to a static value of Treatment.\textit{NONE}, and vice versa. Other private members include the probability of effectiveness (float 0.0 to 1.0), and effectiveness (also float 0.0 to 1.0). If a treatment's probability of effectiveness is 0.5, and its effectiveness is 0.8, then there exists a 50\% chance that the treatment will alleviate 80\% of the active impairment(s); conversely, there exists a 50\% chance the treatment will have no effect.

This script also offers an assortment of public methods to do things like retrieve the treatment's current cost or its current wait time cost, determine if the treatment is effective (determined only once and then cached), or to check if the treatment has been obtained already. \newline \newline
\href{https://bit.ly/2JDPChz}{\textbf{PillManager.cs}}: TODO 

\subsubsection*{Receiving Treatment} % how do they trigger an obtain attempt, what happens on a successful attempt


\subsubsection*{Instructions} % Tutorial framework, how it's coded -> beginning + mid-way via limbo / treatments
\href{https://bit.ly/2JFHEED}{\textbf{Instruction.cs}}: A simple class to encapsulate the two components of any instruction: a string \textit{message}, and a float \textit{displayDuration} (seconds). Instruction objects can be passed into later-described methods in order to display their respective messages for the desired amount of time. \newline \newline
\href{https://bit.ly/2TzLE9i}{\textbf{InstructionManager.cs}}: Offers methods to display instructions on the participant's HUD. When these methods are called, a grey translucent panel is enabled in front of the participant's eyes, and the corresponding instruction text is placed on top. Once an instruction is set and is being displayed, this class is also responsible for disabling the instruction text and panel once the message's display duration has expired. This script is meant to be attached to an empty game object within the scene, and requires instruction text and panel game objects to be attached to it through the Unity editor. \newline \newline
\href{https://bit.ly/2UhmSzq}{\textbf{SimManager.cs : \textit{limbo (Instruction [])}}}: This method is to be used when instructions are required in the midst of the simulation, rather than during the tutorial day, \'Day Zero\'. The programmer must first define an array of \texit{Instruction} objects, and then pass those into the \textit{limbo} method. Once the method is called, the simulation will enter the \textit{LIMBO} state. In this state, the user is not able to complete the task or interact with the environment in any way. Instead, the grey translucent instruction panel will be enabled, and each instruction in the array will be displayed sequentially. Once all instructions have been displayed for their respective durations, the \texit{exitLimbo()} method should be called so that the simulation will resume in the \textit{RUNNING} state. \newline \newline
\href{https://bit.ly/2Yooc2m}{\textbf{GoalMarker.cs}}: A script that defines the behaviour of a point of interest marker meant to accompany a given instruction. The marker will move smoothly up and down, keeping a constant X and Z position within the scene. It requires the programmer to set the upper and lower Y bounds, as well as how fast it should move (all public float variables). This script is designed to be attached to marker game objects within the scene - if the object is activated, it will always exhibit the behaviour defined in this script. \newline \newline
\href{https://bit.ly/2uwAQ1u}{\textbf{InstructionTrigger.cs}}: There are different events that should trigger the next instruction set in a sequence; for example, a certain time duration having elapsed, some threshold being met, or the participant (or some other object) physically moving to a specified location within the scene. This script allows the latter, and is meant to be attached to an invisible game object within the scene (this object must have a collider that has \textit{isTrigger} enabled). In order for the script to work, the following public members must be assigned through the Unity editor: the SimManager, the next Tutorial Step to advance to (use TutorialStep.NULL unless multiple InstructionTrigger objects are being used for one instruction and could try to advance the tutorial step at the same time; in this case, it is advisable to explicitly define which step to advance to such that no duplicate advances are made) and the object to detect collisions with (such as the participant headset, or the container). Optionally, the programmer may also attach a game object to the \texit{destroyOnTrigger} member - this object will be destroyed whenever the aforementioned collision takes place. This is useful when a Goal Marker is being used in conjunction with an instruction. Once the instruction is no longer needed, such as when the participant moves to a certain location, the Goal Marker is also no longer needed and should be destroyed. 


\subsubsection*{User Interfaces}
\href{https://bit.ly/2G9SRs1}{\textbf{MultiDayUIUpdate.cs}}: Responsible for updating the Wall UI components with time remaining, current day, current wage, and earnings across multiple days of the experiment. Its behaviour depends on how many days are being used for the experiment; notably, the layout will display at most four days worth of earnings. If more than four days are used in the experiment configuration, then the script will display earnings for the four most recent days. In order to facilitate this main functionality, the script has public methods, which should be called only by SimManager, to set the current day, total number of days, among other critical fields. \newline \newline
\href{https://bit.ly/2VHtW5B}{\textbf{TransitionMessage.cs}}: When in the transition state, this script will populate text fields to indicate to the participant how much time remains until the next day of the simulation. As well, it will tell the user whether or not the next day is a full health day, or an impaired day. \newline \newline


\subsection*{Custom Configuration} % Aaska's stuff -> using the config file/parsing


\subsection*{Persistence} % How do we track data / what do we track / how do we print it out
\href{https://bit.ly/2Fhq8B2}{\textbf{ParticipantData.cs}}: Static class populated after the participant presses the start button on the welcome screen. The participant's name and sensitivity information is gathered from the input form, and kept inside public static variables \textit{name} (string), \textit{nauseaSensitive} (bool), and \textit{claustrophicSensitive} (bool). These values can be accessed at any point during runtime. \newline \newline
\href{https://bit.ly/2WgL6qR}{\textbf{PopulateParticipantData.cs}}: Facilitates the population of the ParticipantData static class. Each field of the welcome screen input form is mapped to GameObject variable within this script. When the confirm button is pressed, the values of each form element are retrieved, and the ParticipantData variables are assigned accordingly. Once this process is complete, this script also loads the main simulation scene, using a \textit{LoadLevel} method call. \newline \newline
\href{https://bit.ly/2OdbwH6}{\textbf{SimPersister.cs}}: Contains definitions for log file naming, output data directories, as well as output data string formats. When instantiated, this class validates the output directory, and creates a new text file for persistence, which is named using the start time and participant name (if available). Once the output file is created, the constructor calls the \textit{writeIntroduction()} method, which summarizes the participant and application information, and prints CSV column headers.

This class offers a public method \texit{persist()} - it takes a number of important simulation metrics as arguments, and persists them into the output file in CSV format.

\end{document}
